\documentclass[aa]{aa}
\bibpunct{(}{)}{;}{a}{}{,} % to follow the A&A style
\usepackage{graphicx}
\usepackage{comment}
\usepackage{txfonts}
\usepackage{stfloats}
\newcommand{\spz}[1] {{\texttt{\textbf{SPZ: #1}}} }
\newcommand{\eh}[1] {{\texttt{\textbf{ EH: #1}}} }
\newcommand{\LSun}{\mbox{${L}_\odot$}}
\newcommand{\MSun}{\mbox{${M}_\odot$}}
\newcommand{\Msun}{\mbox{${M}_\odot$}}
\newcommand{\RSun}{\mbox{${R}_\odot$}}
\newcommand{\MJup}{\mbox{${M}_{\rm Jup}$}}

\def\apgt{\ {\raise-.5ex\hbox{$\buildrel>\over\sim$}}\ }
\def\aplt{\ {\raise-.5ex\hbox{$\buildrel<\over\sim$}}\ }
\def\lteq{\ {\raise-.5ex\hbox{$\buildrel<\over-$}}\ }

\newcommand{\jumbo}{\mbox{JuMBO}}
\newcommand{\jumbos}{\mbox{JuMBOs}}

%-----------------------------------------------------------------------

\begin{document} 

   \title{The primoridal origin of Jupiter mass Binary Objects}
%   \author{E. Hochart\inst{1}
%          \and
%          S. Portegies Zwart\inst{1}
%          }
   \institute{
             Leiden Observatory, University of Leiden, 
             Niels Bohrweg 2, 2333 CA Leiden\\
             \email{hochart@mail.strw.leidenuniv.nl}
             \email{spzstrw.leidenuniv.nl}
             }
   \date{Received XXXXX; accepted XXXXXX}

%--------------------------------------------------------------------
  \abstract
  % context heading (optional)
   {}
  % aims heading
   {}
  % methods heading
   {}
  % results heading
   {}
  % conclusions heading (optional)
   {}
   \keywords{gravitational waves -- stars: black holes -- galaxies: nuclei -- stars: kinematics and dynamics -- black hole physics
            }

   \maketitle
   
%-------------------------------------------------------------------
\section{Introduction}

Recently \cite{2023arXiv231001231P} reported on the discovery of 42
jupiter-mass binaries in the direction of the Trapezium cluster.
Their component masses are between 0.6\,\MJup\, and 14\,\MJup\, with
projected separations between 25\,au and 380\,au.  The averaged
observed values are $d=200\pm109$\,au, $\langle M\rangle =
4.73\pm3.48$\,\MJup, and $\langle M\rangle = 2.81\pm2.29$\,\MJup.
Two of these objects have a nearby tertiary jupiter-mass companion,
and they found an additional population of 540 single objects in the
same mass range. This discovery initiates the discussions on their
origin and surviveability in a clustered environment.

Jupiter-mass free floaters have been found before, but they are
generally isolated or with small (few au) separations.
\cite{2021ApJS..253....7K}.  Known interstellar jupiter-mass binary
objects include
\begin{itemize}
  \item[$\bullet$] 2MASS J11193254-1137466 AB: a $5$
to 10\,\MJup\, primary in a $a=3.6\pm0.9$\,au orbit \cite{2017ApJ...843L...4B}.
  \item[$\bullet$] WISE 1828+2650: a 3 to 6\MJup\, primary with a
    5\,\MJup\ companion in an $\apgt 0.5$\,au orbit
    \cite{2013ApJ...764..101B}.
  \item[$\bullet$] WISE J0336-014: a $8.5$ to $18$\,\MJup primary with
    a $5$ to $11.5$\,\MJup\, companion in a $0.9^{+0.05}_{-0.09}$\,au
    orbit \cite{2023ApJ...947L..30C}.
  \item[$\bullet$] 2MASS J0013-1143 discovered by
    \cite{2017AJ....154..112K} and suspected to be a binary by
    \cite{2019A&A...629A.145E}.
\end{itemize}

Star formation, from the collapse of molecular clouds through
gravitational instability, generally are expected to lead to objects
considerably more massive than Jupiter \citep{Low1976, Boyd2005}.  As
a consequence, the large population of jupiter-mass free-floaters was
considered to result from the ejected planets from dynamically
unstable planetary systems.  Several studies considered the
possibility of planetary systems losing outer-planets in dynamical
interactions in dense stellar systems (see i.e.,
\citet{1996Sci...274..954R,2015MNRAS.453.2759Z, 2017MNRAS.470.4337C,
  2019MNRAS.489.2280F, 2019A&A...624A.120V}), but they focus on the
ejection of single planets, not binaries.  Their origin through
dynamical phenomena gets further complicated by the tendency for lower
mass planets to be more prone to ejections \citep{Ford2001,
  2013MNRAS.433..867H,2019MNRAS.489.2280F,2020MNRAS.497.1807S}.

Alternative to forming in situ (which we call scenario ${\cal SF}$),
one can naively imagine three mechanisms to form jupiter mass binary
objects (\jumbos) \cite{2023arXiv231006016W} argued that these
binaries could be explained from planetary systems of which the outer
two planets are stripped by a passing star in a close encounter. The
two ejected planets would lead to a population of free floating
planets, but also explain the observed population of \jumbos.  We call
this scenario ${\cal PP}$.

Alternatively one could imagine \jumbos to result from planet-moon
pairs orbiting some star that is ejected to become a jumbo.  We call
this scenario ${\cal PM}$).

Finally, one could imagine that with a sufficiently large population
of free-floating jupiter-mass objects could lead to a population of
jumbos by dynamical capture of one jupiter by another.  We call this
scenario ${\cal FFC}$. A similar scenario was proposed
\cite{2010MNRAS.404.1835K,} for explaining very wide stellar pairs,
but the model was adapted to account for wide planetary orbits
\cite{2012ApJ...750...83P}.

We start by discussing some fundamental properties of the
environmental dynamics, followed by a description of numerical
simulations to characterize the parameters of the acquired jumbos and
the resulting occurrence rates.

%--------------------------------------------------------------------
\section{The dynamical characterization of \jumbos}

The \jumbos\, discovered by \cite{2023arXiv231001231P} were location
in the Trapezium cluster. Assessing the cluster dynamics we base our
analysis on the analysis carried out by \cite{2016MNRAS.457..313P}.
He determined cluster parameters by numerical modeling of the
distribution of disk sizes observed in the Trapezium cluster, and
concluded that this distribution is best reproduced for a cluster
containing some 2500 stars with a total mass of $\sim 880$\,\MSun\,
and a half-mass radius of around 0.5\,pc. The results were only
consistent with the observations if the initial cluster density
distribution represented a fractal dimension of 1.6, and was
inconsistent with a Plummer \cite{1911MNRAS..71..460P} distribution.
For consistency with earlier studies, we perform our analysis for
Plummer as well as for a fractal (with fractal dimension 1.6)
distributions.

Adopting a Plummer distribution of the Trapezium cluster would
indicate a core radius of about 0.34\,pc with a core mass of
250\,\MSun, which results in a velocity dispersion of $\sim
0.73$\,km/s. With a mean stellar mass in the cluster core of
1\,\MSun\, the unit of energy expressed in the kinematic temperature
kT becomes $\sim 8 \cdot 10^{42}$\,erg.

Jumbos are found in the mass range of about 0.6\,\MJup\, to
14\,\MJup\, and with a projected separation of 25\,au\, and $\sim
380$\,au. For clarity we adopt here that the observed range in
projected distances between the two jupiter-mass objects is consistent
with an orbital separation, and express distances in terms of
semi-major axis instead.

To first order, the binding energy of jumbos then ranges between $\sim
5\cdot 10^{37}$\,erg and $1.4\cdot 10^{41}$\,erg, or at most $\sim
0.02$\,kT. Which makes them soft upon an encounter with a cluster
star.

The hardest \jumbo, composed of two 14\,\MJup\, planets in a 25\,au
orbit would be hard for another encountering object of less then
$17\,M_{\rm Jupiter}$.  For an encountering 1\,\MJup\, object a 25\,au
orbit would be hard only if the two planets are about three times as
massive a Jupiter.  The majority of jumbos in the trapezium cluster
are then still soft for any encountering free floating planet, but
hard if their orbits are tighter, or the encountering free floating
planet has low mass.

The nature of these soft encounter probably soften these binaries even
further, whereas an occasional soft encounter may harden the jumbo.
An low impact-parameter encounter with a star will tend to ionize any
of the observed jumbos.  Independent of how tight the orbit.  Jumbos,
therefore, are expected to be relatively short lived, and dissociate
upon a close encounter with any other cluster member.

to further understand the dynamics of the jumbos in a clustered
environment, and to study the efficiency of the various formation
scenario's we perform $N$-body calculations of a Trapezium-like star
cluster with a population of jupiter-mass objects in various initial
configurations.

%--------------------------------------------------------------------
\section{Model calculations}

For each of our proposed models, ${\cal SF}$ (in situ formation of
jumbos), ${\cal PP}$ (as outer orbiting planets), ${\cal PM}$ (as bound
planet-moon pair orbiting a star), and ${\cal FFC}$ (as mutual
recapture of free-floaters) we perform a series of $N$-body
simulations with properties consistent with the Trapezium cluster.

Each cluster starts with 2500 single stars taken from a Kroupa mass
function \cite{2002MNRAS.336.1188K} between 0.08\,\MSun\, and
$30$\,\MSun\, distributed either in a Plummer sphere or a fractal
distribution with a fractal dimension of 1.6 in virial equilibrium.  We
run three models for each set of initial conditions, with a virial
radius of 0.25\,pc, 0.5\,pc and 1.0\,pc, called model R025, R05 and
R10, respectively.  We ignore stellar evolution, as well as the
tidal field of the Galaxy.

For each of our proposed models, we initialize a population of single
or binary jupiter-mass objects. The single (and primaries in
primordial jumbos) are selected from a power-law mass function between
0.6\,\MJup\, and 14\,\MJup, which is consisted with the observed mass
function \cite{2023arXiv231001231P}.

For the models with free-floating jupiter-mass objects, scenario
${\cal FFC}$, we sprinkle the single planets in the cluster potential
as single objects using the same initial distribution function as Watt
we used for the single stars.  These models were run with
1200\,jupiter-mass objects, but we performed additional runs with
$10^3$ and $10^4$ free floaters.

primordial \jumbos\, are initialized with semi-major axis with a flat
distribution between 10\,au and $10^3$\,au, an eccentricity from the
thermal distribution between 0 and 1, and a mass ratio (also from the
thermal distribution) between 0.2 and 1.  The binary is subsequently
rotated to a random orientation. We typically start with 1200 single
or 600 jupiter-mass pairs.

Isolated binaries, for scenario ${\cal SF}$, are subsequently sprinkled
in the cluster potential as single objects using the same initial
distribution function as wat we used for the single stars.

For scenario ${\cal PM}$ we put the bound planet-moon pair in orbit
around a star.  The orbit of the planet-moon pair is circular and with
a random orientation at a distance from the star such that the
planet-moon's orbital separation is one-third of it's Hill radius.
This guarantees a stable planet-moon pair in orbit around the selected
star.

We selected the star to host such a planet-moon pair from the first
150 stars lower in mass than the mean stellar mass, and the first 150
stars more massive than the mean.

For scenario ${\cal PP}$, we select the same planet masses as for the
primordial \jumbos\, except that we have them orbiting one of the
selected stars as a hierarchical planetary system. The distance from
the first planet $a1$ and the second planet $a2$ (such that $a2>a1$)
are selected according to various criteria. The outer orbit, $a2$, we
typically chose to be ten times larger than the inner planet's Hill
radius, but we also perform simulations with five times and twice the
Hill radius (we call them model rH10, rH05 and rH02, respectively).

We perform an additional series of runs with pre-specified orbital
separations for the two planets $a1$ and $a2$, to follow the model
proposed in \cite{2023arXiv231006016W}.

Each run was repeated 10 times to deal with potential statistical
fluctuations, but we run 40 initiations of models ${\cal PP}$\_R025.

Each simulation is stopped at an age of 1\,Myr, after which we study
the population of free floating jupiter-mass objects and the
population of \jumbos.

To summarize, we performed the following model calculations:
\begin{itemize}
\item[${\cal PP}$]: as outer orbiting planets
\item[${\cal PM}$]: as bound planet-moon pair orbiting a star
\item[${\cal FFC}$]: free floating single planets.
\item[${\cal SF}$]: in situ formation of jumbos
\end{itemize}

\section{Results}

\subsection{Model ${\cal PP}$}

In scenario ${\cal PP}$, we follow the dynamical evolution of 1900
single stars, and 600 stars what are orbited by two planets.  According
to \cite{2023arXiv231006016W}, jumbos form naturally upon a dynamical
encounter between the planetary systems and a passing star.  In
table\,\ref{Tab:model_PP} we summarize the results of model ${\cal
  PP}$.

\begin{table}
 \caption{...}
 \label{Tab:model_PP}
 \centering 
 \begin{tabular}{lrrrrrrr}
 \hline\hline
model &$R_{\rm vir}$ & $N/R_{\rm run}$ & $\langle M \rangle$ & $\langle q \rangle$ & $\langle a \rangle$ & $\langle e \rangle$ \\
        \hline \vspace{-0.75em}\\
 Plummer & 0.25 & 0.4 &  7.35	&2.35&	2388 &0.70	\\
 Plummer & 0.50 & 0.7 & 6.76    & 2.44 & 2196 & 0.86 \\
 Plummer & 1.00 & 0.4 & 3.77    & 1.67 & 1507 & 0.28\\
 Fr 1.6  & 0.25 & 0.0 &   ---   & ---  &  --- & --- \\			
 Fr 1.6  & 0.50 & 0.1 & 3.8     & 1.9 & 118 & 0.99\\
 Fr 1.6  & 1.00 &0.2  & 8.3     & 5.8 & 147 & 0.94 \\ 
 \hline
 \end{tabular}
\end{table}

These calculations fail to reproduce the number of observed \jumbos\,
by more than an order of magnitude, and the eventual orbital parameters
are inconsistent with the observed population.  The fractal model with
a virial radius of 1\,pc, produces 0.2\,jumbos\ with a mean orbital
separation of 147\,au. The most abundant Plummer model still produces
less than 1\,\jumbo\, per cluster, and in a much wider orbit than
expected based on the observations.

To further explore the failure of model ${\cal PP}$, we perform an
additional series of simulations with pre-determined inner and outer
orbital separations $a1$ and $a2$ using the Plummer distribution with
virial radii of 0.25\,pc, 0.50\,pc and 1.0\,pc for the stars.
According to \cite{2023arXiv231006016W}\, the eventual orbital
separation of the \jumbo\, would be consistent with the difference in
orbital separation between the two planets when orbiting the star. For
this reason we perform an additional series of runs with a mutual
separation $a2-a1 = 200$\,au, expecting those to lead to consistent
results in comparison with the observations.  The results of these
simulations are presented in figure\,\ref{Fig:fjumbos_from_PP}.

We observed that the jumbo formation efficiency peaks for an orbital
separation $a1 \sim 800$\,au to $a1 \sim 1600$\,au. In those cases the
secondary orbits range from $a2 \sim 1000$\,au to $a2 \sim 1800$\,au.
The total number of jumbos\, however, remains small, at most 10 jumbos
are produced per run, and the fraction of free floating single planets
is much higher than expected based on the observations.

The failure of model ${\cal PP}$, can also be understood from
carefully reading \cite{2023arXiv231006016W}. Their highest cross
section is achieved for the orbital velocity of the inner-most planet
as fraction of the typical encounter velocity $v1/v_{\rm disp} \sim
0.8$. With a cluster velocity dispersion of $\sim 0.8$\,km/s, the
orbital velocity roughly 1\,km/s. Around an 1\,\MSun\, star such a
velocity is obtained, assuming a Kepler orbit, at an orbital
separation of 800\,au. It turns out, that the results of the cross
section calculations performed by \cite{2023arXiv231006016W} are
consistent with our direct N-body simulation of the entire cluster,
but that the adopted initial orbital separation is too wide in
comparison with a realistically population of inner planetary orbits
for jupiter-mass planets.  We agree that the observational selection
effect of finding $\apgt 800$\,au jupiter-mass planets is quite
severe, but we consider it unrealistic to have 600 out of 2500 stars
to be orbited by such wide planetary systems. In particular, when once
considers the small sizes of the observed disks in the Trapezium
cluster, which today are all smaller than 400\,au
\citep{2005A&A...441..195V}.

\begin{figure}
    \centering
        \includegraphics[width=.91\columnwidth]{figures/fig_fjumbos_from_psystems.pdf}
        \caption{The number of jumbo's produced in model ${\cal PP}$,
          as fraction of the number of free floating planets for
          various simulations starting with a Plummer sphere with a
          virial radius of 0.5\,pc.  The bullet points along each line
          correspond with the adopted orbital separation of the two
          planets ($a_1$ and $a_2$).  The red symbols indicate an
          average orbital separations for the jumbos between 25\,au
          and 380\,au.  The black symbols are outside this regime.
          The symbol sizes give the number of jumbos (see top right
          for scaling) in the particular simulation.  }
         \label{Fig:fjumbos_from_PP}
\end{figure}

\subsection{Model ${\cal PM}$}


\subsection{Model ${\cal FFC}$}

\subsection{Model ${\cal SF}$}

\begin{figure}
    \centering
        \includegraphics[width=.91\columnwidth]{figures/sem_axis_Fractal_FF.pdf}
        \caption{}
         \label{Fig:Fr_semimajor_axis}
\end{figure}

\begin{figure}
    \centering
        \includegraphics[width=.91\columnwidth]{figures/mass_distr_Plummer_rvir0.5.pdf}
        \caption{}
         \label{Fig:Plummer_massfunction}
\end{figure}
\begin{figure}
    \centering
        \includegraphics[width=.91\columnwidth]{figures/mass_distr_Fractal_rvir0.5.pdf}
        \caption{}
         \label{Fig:fractal_massfunction}
\end{figure}



  


\section{Discussion}


They calculate the rate by means of 4-body scattering experiments, in
which a star with two equal-mass planets with semi-major axes $a_1$
for the inner and $a_2$ for the other planet, encounters a single
star. Their largest cross section of roughly $a_1^2$ is obtained if
the encounter velocity $0.8v_\star/v_1$. For an encounter at the
cluster's velocity dispersion, the inner planet would then have a
orbital separation of about 900\,au around a 1\,\MSun\, star.

The orbital separation of the eventual jumbo would then be comparable
to the difference in orbital separation between the two planets
($a_{\rm Jumbo} \simeq a_2-a_1$).

The model in which \jumbos\, form a natural byproduct of the low-mass
end of the star-formation process




\subsection{\jumbos\, as former planet-moon pairs}

According to \cite{2023arXiv231015603C}, jumbos form naturally upon a
dynamical encounter between two stars one of which with orbited by a
binary planet or planet-moon system.

Both calculations \cite{2023arXiv231006016W} and
\cite{2023arXiv231015603C} adopt scattering experiments to determine
the formation rate of jumbos from their adopted initial conditions.


%--------------------------------------------------------------------
\section{Discussion}

\cite{2023arXiv231015603C} argued that \jumbos potentially originate
from tilted circum-binary planets. Formed as a --sort-of-- planet-moon
system in a wide orbit around a star, that is strippied from the host
star by the cluster potential or a relatively wide encounter with
another star. 



%--------------------------------------------------------------------
\section{Conclusion}

\section*{Acknowledgements}

Veronica Saz Ulibarrena, Shuo Huang, Maite Wilhelm, Brent Maas
    
\input /home/spz/Latex/lib/bib/references
    
\end{document}

%%%%%%%%%%%%%%%%%%%%%%%%%%%%%%%%%%%%%%%%%%%%%%%%%%%%%%%%%%%%%%
%-------------------------------------------------------------------
% END OF TEXT
%-------------------------------------------------------------------
%%%%%%%%%%%%%%%%%%%%%%%%%%%%%%%%%%%%%%%%%%%%%%%%%%%%%%%%%%%%%%

Pl
Rvir N/run <M>	<m>	<a>	<e>	Nruns
0.25 0.4   7.35	2.35	2388	0.70	40
0.50 0.7   6.76	2.44	2196	0.86	10
1.00 0.4   3.77	1.67	1507	0.28	10

Fr1.6 
Rvir N/run <M>	<m>	<a>	<e>	Nruns
0.25 0.0				10
0.50 0.1   3.8	1.9	118	0.99	10
1.00 0.2   8.3	5.8	147	0.94	10

Pl with Rvir= 0.25pc with various secondary planet orbits (rHill)
rH N/run <M>	<m>	<a>	<e>	Nruns
10 0.4   7.35	2.35	2388	0.70	40
5  2.6	 8.32	2.27	1895	0.78	10
2  8.6	 6.60	2.89	1448	0.82	10

Pl with Rvir= 0.25pc at various times (t)
t   N/run  <M>	<m>	<a>	<e>	Nruns
1.0 0.40   7.35	2.35	2388	0.70	40
0.5 0.15   5.43	1.58	4400	0.67	40

Pl with Rvir= 0.25pc rH=5, q=sqrt() 
t   N/run  <M>	<m>	<a>	<e>	Nruns
1.0 0.8	   2.50 1.36   2716.31   0.777  10
